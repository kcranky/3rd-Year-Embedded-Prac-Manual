\setcounter{section}{-1}
\section{Prac 0 - Setting up and Connecting to Your Pi}
The Raspberry Pi is a powerful computing platform. While it can be used as a lightweight computer, its true strengths are found when it is used as an embedded system or edge processor.\footnote{The term “edge computing” has become popular nowadays to refer to methods of bringing the computing power and its memory and peripherals closer to the location where these are needed (you could say that edge computing is the antithesis of cloud computing).}  We will be using the Pi throughout the rest of the course in both practicals and the mini project.  Before you attend the first tutorial session is is required that you setup your Pi so it is ready for operation.  In addition, if you are unfamiliar with Linux, Git, ssh, and bash we strongly recommend you go through the additional exercises as outlined in the lab handbook.

\subsection{Required before attending the first lab session: Setup your Pi }
Before arriving at the first practical lab session of this course you are required to:
\begin{enumerate}
    \item Setup your Pi. This includes installing the image, getting it to work over SSH and enabling VNC and other services listed in the installation section of the handbook.
    \item Have created an account on a git based cloud repository hosting service that will allow you to collaboratively work with other team members.
\end{enumerate}

If you have trouble completing the following you should attend the hotseat on friday the 19th of July to get help before the first lab session.
