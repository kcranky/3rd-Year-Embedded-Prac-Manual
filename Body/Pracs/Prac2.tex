\newpage
\section{Prac 2 - Python vs. C}
\label{sec:Prac2}

\subsection{Overview}
This practical will show you the importance of C for embedded systems development. 

\subsection{Pre-prac Requirements}
This section covers what you will need to know before starting the practical.
\begin{itemize}
    \item Cross compilation
    \item Makefiles
    \item Introduction to benchmarking concepts (not looking at only one benchmark, hot boards, etc)
    \item Introduction to cache warming and good testing methodology (multiple runs, wall clock time, speed up)
\end{itemize}

\subsection{Outcomes}
You will learn about the following aspects:
\begin{itemize}
    \item Heterodyning
    \item Benchmarking
    \item Latex
    \item Bit widths (too much?)
    \item Compiler flags (too much?)
    \item Instruction sets (will be required for floating and compiler flags eg \$cat /proc/cpuinfo)
\end{itemize}

\subsection{Deliverables}
At the end of this practical, you must:
\begin{itemize}
    \item Submit a 3 page report in IEEE-Conference style detailing your investigation
\end{itemize}

\subsection{Hardware Required}
\begin{itemize}
    \item Raspberry Pi
    \item SD Card
    \item Ethernet Cable
\end{itemize}

\subsection{Further Instructions}
\textbf{Instructions to tutors:}
\begin{itemize}
    \item Further develop the prac overview to give a short blurb on heterodyning (it's use, etc)
    \item Develop C and Python code for heterodyning
    \item Include the guides for cross compilation as listed in the appendix
    \item Decide on an independent variable for testing the implementation (bit width maybe? 16 vs 32 vs 64 bit floating point by setting C compiler flags? Or decide if just testing C and python is enough)
    \item Write up the walkthrough section of this practical and adapt any other sections of this prac
    \item Link to Overleaf and a guide on how to get the IEEE conference template (new project -\textgreater  from template -\textgreater  etc)
\end{itemize}


Not sure if the following  is too much but I've included it anyway in case the tutors decide it's doable given the nature of the prac:\\
Prac 2 should be be a mathematical benchmark comparison, using floating vs fixed point math, dealing with compiler optimizations for C and hardware level support for the RPi. Students are also exposed to instruction sets (\$cat /proc/cpuinfo), compiler flags (-mfp16-format=ieee, setting HW floating point compiler directives such as fpv4, vfpv3, vfpv3xd, etc).

