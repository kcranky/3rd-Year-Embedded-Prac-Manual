\newpage
\section{Prac 4 - SPI and Threading}
\subsection{Overview}
While the Pi 3B+ does have an audio jack, it uses a purely PWM (pulse width modulation) based implementation for audio. Audiophiles among you may know that this is not a great way of playing audio. You can read more about the Raspberry Pi and it's audio jack \href{https://hackaday.com/2018/07/13/behind-the-pin-how-the-raspberry-pi-gets-its-audio/}{here}.

To improve audio quality, we can use a DAC (digital to analog converter). In this practical, you will pass sampled audio data over SPI to a 10 bit DAC for playback. 

\subsection{Pre-prac requirements}
This section covers what you will need to know before starting the practical.
\begin{itemize}
    \item Read the \href{http://ww1.microchip.com/downloads/en/DeviceDoc/22248a.pdf}{MCP4911 Datasheet}
\end{itemize}

\subsection{Outcomes}
You will learn about the following aspects:
\begin{itemize}
    \item SPI
    \item Threading
\end{itemize}

\subsection{Deliverables}
At the end of this practical, you must:
\begin{itemize}
    \item Demonstrate your working implementation to a tutor
    \item Submit your code on Vula
\end{itemize}

\subsection{Hardware Required}
\begin{itemize}
    \item Raspberry Pi
    \item SD Card
    \item Ethernet Cable
    \item Breadboard
    \item Jumper wires
    \item X x Buttons (Depends on number of samples)
    \item MCP4911 DAC Chip
    \item Speaker to play sound (Supplied by lab?)
\end{itemize}


\subsection{Further Instructions}
\begin{itemize}
    \item If you didn't in Prac 0, enable SPI in raspi-config
\end{itemize}