\newpage
\section{Prac 4 - I2C and PWM}
\label{sec:Prac4}
\subsection{Overview}
Something about why loading scripts on Pi on startup is useful. Something about RTCs and why they are useful (especially considering the Pi doesn't have one)

\subsection{Pre-prac requirements}
This section covers what you will need to know before starting the practical.
\begin{itemize}
    \item I2C
    \item PWM
    \item Knowledge about BASH as learnt in previous pracs
\end{itemize}

\subsection{Outcomes}
You will learn about the following aspects:
\begin{itemize}
    \item I2C
    \item Real Time Clocks
    \item Starting a script on boot on the Raspberry Pi
\end{itemize}

\subsection{Deliverables}
At the end of this practical, you must:
\begin{itemize}
    \item Demonstrate your working implementation to a tutor
    \item Submit your code on Vula
\end{itemize}

\subsection{Hardware Required}
\begin{itemize}
    \item Raspberry Pi
    \item SD Card
    \item Ethernet Cable
    \item Breadboard
    \item Jumper wires
    \item 2 x Buttons
    \item RTC
    \item 11 LEDs (4 for hours, 6 for minutes, 1 for seconds)
    \item 11 Resistors for said LEDs
\end{itemize}


\subsection{Further Instructions}
\textbf{Instructions to tutors:}
\begin{itemize}
    \item Flesh out all the sections in the guide
    \item Complete this practical and write instructions for students
    \item The time from the RTC should be displayed in binary format on the LEDs
    \item The "seconds" LED should be at it's dimmest when the minute starts, and at it's brightest just before the minutes value increases (ie it should be min brightness at 0 seconds, and max brightness at 59.99 seconds)
    \item One button should increase the hours value, another button increase the minutes value
\end{itemize}