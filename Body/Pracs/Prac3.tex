\newpage
\section{Prac 3 - SPI and Threading}
\label{sec:Prac3}
\subsection{Overview}
While the Pi does have an audio jack, it uses a purely PWM (pulse width modulation) based implementation for audio. Audiophiles among you may know that this is not a great way of playing audio. You can read more about the Raspberry Pi and it's audio jack \href{https://hackaday.com/2018/07/13/behind-the-pin-how-the-raspberry-pi-gets-its-audio/}{here}.

To improve audio quality, we can use a DAC (digital to analog converter). In this practical, you will pass sampled audio data over SPI to a 10 bit DAC for playback. 

\subsection{Pre-prac requirements}
This section covers what you will need to know before starting the practical.
\begin{itemize}
    \item Interrupts
\end{itemize}

\subsection{Outcomes}
You will learn about the following aspects:
\begin{itemize}
    \item SPI
    \item Threading
\end{itemize}

\subsection{Deliverables}
At the end of this practical, you must:
\begin{itemize}
    \item Demonstrate your working implementation to a tutor
    \item Submit your code on Vula
\end{itemize}

\subsection{Hardware Required}
\begin{itemize}
    \item Raspberry Pi
    \item SD Card
    \item Ethernet Cable
    \item Breadboard
    \item Jumper wires
    \item X x Buttons (Depends on number of samples)
    \item MCP4911 DAC Chip
    \item Speaker to play sound (Supplied by lab?)
\end{itemize}


\subsection{Further Instructions}
\textbf{Instructions to tutors:}
\begin{itemize}
    \item Flesh out all the sections in the guide
    \item Complete this practical and write instructions for students
    \item You must be able to press a button to play audio. 
    \item Transfer to the DAC should happen on a second thread
    \item If something is playing and another button is pressed, the new audio should be played (ie the current playing audio should be replaced by the new file)
\end{itemize}