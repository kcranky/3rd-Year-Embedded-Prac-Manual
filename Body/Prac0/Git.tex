\section{Git}
Throughout the course you will be required to write and submit code and other files to an online git repository.  Git is a popular version control application that allows you to keep a record of changes you have made to text based files over time.  Additionally, by using a version controlled repository that is accessible to other collaborators, it provides a powerful way for many people to collectively build a larger system.  Almost all open source projects use a version control system to enable them to work with anyone else around the world in a structured and organised manner.  There are many platforms that offer free hosting services where you are able to share repositories, Github, and Gitlab are 2 such options.  Making an account on either of these (or a similar alternative) allows you to: (1) keep track of your own projects and the changes you make to them, (2) work collaboratively with team mates if you choose to share your repositories with them, and (3) is an increasingly common way of sharing a portfolio of your prior work with potential employers.

To setup your account in either of these follow the tutorials supplied here:
- GitHub: Sign up here https://github.com/join

Once you have created an account you need to also install git on your own computer.  If you are using a lab computer git is already installed.  If you are using your own computer follow the instructions for your operating system here \href{https://carpentries.github.io/workshop-template/#git}{https://carpentries.github.io/workshop-template/#git}

Before arriving at Prac 1, make sure you have created a demo repository on a computer and pushed it to the cloud using the instructions found in Section \ref{app:Git}.

