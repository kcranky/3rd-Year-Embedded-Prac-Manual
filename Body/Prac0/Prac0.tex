\setcounter{section}{-1}
\section{Prac 0 - Setting up and Connecting to Your Pi}
The Raspberry Pi is a powerful computing platform. While it can be used as a lightweight computer, it's true strengths are found when it is used as an embedded system or edge processor.\footnote{The term “edge computing” has become popular nowadays to refer to methods of bringing the computing power and its memory and peripherals closer to the location where these are needed (you could say that edge computing is the antithesis of cloud computing).}  We will be using the Pi throughout the rest of the course in both practicals and the mini project.  Before you attend the first tutorial session is is required that you setup your Pi so it is ready for operation.  In addition, if you are unfamiliar with Linux, Git, ssh, and bash we strongly reccommend you go through the additional excerises as outlined bellow.

\subsection{Required before attending the first lab session: Setup your Pi }
Before arriving at the first practical lab session of this course you are required to:
\begin{enumerate}
    \item Setup your Pi
    \item Have created an account on a git based cloud repository hosting service that will allow you to collaboratively work with other team members.
\end{enumerate}

If you have trouble completing the following you should attend the Friday hotseat to get help before the fist lab session.

\subsection{Setting up your Pi}
In order to use the Pi you need to install an operating system on it and setup its networking.  The following will lead you through installing Raspberian as an operating system.  Raspberian  is a fork of the Debian distribution of the open source operating system Linux.  You will then configure the networking settings in Raspberian to allow you to access the Pi remotely using SSH.  

It is assumed that the user is running Windows, but the process will be similar for users of any operating system.

\subsubsection{Prerequisites}
\label{sec:Prereqs}
Download the following software applications and install them as indicated on their webpages
\textbf{TODO: These resources are also available on Vula.}
\begin{itemize}
    \item Latest Rasbian Stretch with Desktop and Recommended Software image, available at\\ \href{https://www.raspberrypi.org/downloads/raspbian}{Raspberry Pi Foundation Website}
    \item PuTTY (the full suite) available at\\ \href{https://www.chiark.greenend.org.uk/~sgtatham/putty/latest.html}{https://www.chiark.greenend.org.uk/~sgtatham/putty/latest.html}. \footnote{Some versions of Windows have the required SSH and SCP packages, but PuTTY has many other tools that are useful.}
    \item SDFormatter, available at\\ \href{https://www.sdcard.org/downloads/formatter_4/eula_windows/index.html}{https://www.sdcard.org/downloads/formatter\_4/eula\_windows/index.html}
    \item Etcher, an image writing tool, available at\\
    \href{https://www.balena.io/etcher/}{https://www.balena.io/etcher/}
\end{itemize}

\subsection{Install and configure Raspberian}
\begin{enumerate}
    \item Insert the SD card into the computer. If you do not have a reader available, speak to a friend or tutor who does.
    \item Write image to SD card\\
        Open Etcher. Select the downloaded zip image, and the SD card, and format. At the end of format, it may read that it failed,  but don't worry. Upon completion, Windows will try to mount partitions on the SD card that it can't read. Just press "Cancel" and then "OK" to the dialog boxes that pop up. The boot partition is the only partition we will be dealing with.
    \item Boot\\
    Insert SDCard into Pi and Plugin a screen and keyboard....
    \item Configure\\
    \begin{itemize}
        \item Open a terminal (by.....) and run raspi-config..... and enable ssh, SPI, I2C, Serial, ....
        \item Create a new user 
        \item Add new user to sudoers
    \end{itemize}
\end{enumerate}


\subsection{Testing}
Once you've completed the installation process, including all relevant network configurations as laid out in \ref{sec:Prereqs}, \ref{process}, \ref{sec:Connectivity-Ethernet}, \ref{sec:Connectivity-ChangeComputerIP} and \ref{sec:SSH}, be sure to test your connection and configuration before the first lab by logging in to the Pi through SSH or VNC.
\subsection{Git}
Throughout the course you will be required to write and submit code and other files to an online git repository.  Git is a popular version control application that allows you to keep a record of changes you have made to text based files over time.  Additionally, by using a version controlled repository that is accessible to other collaborators, it provides a powerful way for many people to collectively build a larger system.  Almost all open source projects use a version control system to enable them to work with anyone else around the world in a structured and organised manner.  There are many platforms that offer free hosting services where you are able to share repositories, Github, and Gitlab are 2 such options.  Making an account on either of these (or a similar alternative) allows you to: (1) keep track of your own projects and the changes you make to them, (2) work collaboratively with team mates if you choose to share your repositories with them, and (3) is an increasingly common way of sharing a portfolio of your prior work with potential employers.

To setup your account in either of these follow the tutorials supplied here:
- GitHub: Sign up here https://github.com/join

Once you have created an account you need to also install git on your own computer.  If you are using a lab computer git is already installed.  If you are using your own computer follow the instructions for your operating system here \href{https://carpentries.github.io/workshop-template/#git}{https://carpentries.github.io/workshop-template/#git}

Before arriving at Prac 1, make sure you have created a demo repository on a computer and pushed it to the cloud using the instructions found in Section \ref{app:Git}.



\subsection{Recommended introductory materials}

\textbf{VNC}