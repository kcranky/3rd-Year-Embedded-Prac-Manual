\setcounter{section}{-1}
\section{Prac 0 - Setting up and Connecting to Your Pi}
The Raspberry Pi is a powerful computing platform. While it can be used as a lightweight computer, it's true strengths are found when it is used as an embedded system or edge processor.\footnote{The term “edge computing” has become popular nowadays to refer to methods of bringing the computing power and its memory and peripherals closer to the location where these are needed (you could say that edge computing is the antithesis of cloud computing).}

In order to set up your Pi as fast as possible, you need to install the required software (primarily everything required to get you running Raspbian on the Pi), create a subnet to which your local machine (PC) and the Raspberry Pi can connct, and enable SSH. Following Sections \ref{sec:Prereqs}, \ref{sec:Process}, \ref{sec:Connectivity-Ethernet}, \ref{sec:Connectivity-ChangeComputerIP} and \ref{sec:SSH} will guide you through this process.

It's also strongly suggested that  you acquaint yourself with the BASH shell. A guide is given in Section \ref{app:UsefulResources}.

There is another option for installation known as NOOBS. This is not used in this course as we'd like to get you to do a full install and configuration of an OS.

% Installation
\subsection{Setting up your Pi}
In order to use the Pi you need to install an operating system on it and setup its networking.  The following will lead you through installing Raspberian as an operating system.  Raspberian  is a fork of the Debian distribution of the open source operating system Linux.  You will then configure the networking settings in Raspberian to allow you to access the Pi remotely using SSH.  

It is assumed that the user is running Windows, but the process will be similar for users of any operating system.

\subsubsection{Prerequisites}
\label{sec:Prereqs}
Download the following software applications and install them as indicated on their webpages
\textbf{TODO: These resources are also available on Vula.}
\begin{itemize}
    \item Latest Rasbian Stretch with Desktop and Recommended Software image, available at\\ \href{https://www.raspberrypi.org/downloads/raspbian}{Raspberry Pi Foundation Website}
    \item PuTTY (the full suite) available at\\ \href{https://www.chiark.greenend.org.uk/~sgtatham/putty/latest.html}{https://www.chiark.greenend.org.uk/~sgtatham/putty/latest.html}. \footnote{Some versions of Windows have the required SSH and SCP packages, but PuTTY has many other tools that are useful.}
    \item SDFormatter, available at\\ \href{https://www.sdcard.org/downloads/formatter_4/eula_windows/index.html}{https://www.sdcard.org/downloads/formatter\_4/eula\_windows/index.html}
    \item Etcher, an image writing tool, available at\\
    \href{https://www.balena.io/etcher/}{https://www.balena.io/etcher/}
\end{itemize}

\subsection{Install and configure Raspberian}
\begin{enumerate}
    \item Insert the SD card into the computer. If you do not have a reader available, speak to a friend or tutor who does.
    \item Write image to SD card\\
        Open Etcher. Select the downloaded zip image, and the SD card, and format. At the end of format, it may read that it failed,  but don't worry. Upon completion, Windows will try to mount partitions on the SD card that it can't read. Just press "Cancel" and then "OK" to the dialog boxes that pop up. The boot partition is the only partition we will be dealing with.
    \item Boot\\
    Insert SDCard into Pi and Plugin a screen and keyboard....
    \item Configure\\
    \begin{itemize}
        \item Open a terminal (by.....) and run raspi-config..... and enable ssh, SPI, I2C, Serial, ....
        \item Create a new user 
        \item Add new user to sudoers
    \end{itemize}
\end{enumerate}

\subsection{}

\subsubsection{Testing}
Once you've completed the installation process, including all relevant network configurations as laid out in \ref{sec:Prereqs}, \ref{process}, \ref{sec:Connectivity-Ethernet}, \ref{sec:Connectivity-ChangeComputerIP} and \ref{sec:SSH}, be sure to test your connection and configuration before the first lab by logging in to the Pi through SSH or VNC.