\section{Prac 2}
This prac serves as an introduction to SPI, Interrupts and Threading. Python is still used as it is easy and can be used to introduce the concepts. 

\textbf{From lab manual}

In the new Practical 2 for the 2019 class, students should use the timer to measure the time between two successive button pushes, and print the time in milliseconds to the display. The time between the two button pushes can then be used to control the period of the PWM signal (in e.g. us accuracy). A C function should be written to calculate the duty cycle of the PWM signal in order to generate a single sinusoid at the output of the LPF. If we have scopes available for testing, that would be great to observe the time-domain output signal. Otherwise, an earphone could be used to hear the PWM DAC circuit output signal after passing the signal through an amplifier.

The new Practical 2 for 2019 should be done in teams of two students.

If students are given DAC SoC components, they could finish the practical in one session. If they are building their own DAC using a PWM signal and a LPF, they may need two practical sessions to finish.