\section{Prac 3 - SPI and Threading}
\label{sec:Prac3}
\textbf{I've just included the prac from last year}
This prac serves as an introduction to SPI, Interrupts and Threading. By the end of the practical you will have a simple environment monitor. The system includes a temperature sensor (MCP9700A), a LDR (1K), a pot (1K), a MCP3008 IC and three switches.

\begin{itemize}
    \item By default, the system continuously monitors the sensors every 500ms using this format:
    \begin{table}[H]
    \centering
    \begin{tabular}{|l|l|l|l|l|}
    \hline
    Time     & Timer    & Pot   & Temp & Light \\ \hline
    10:17:15 & 00:00:00 & 0.5 V & 25 C & 10 \% \\ \hline
    10:17:20 & 00:00:05 & 1.5 V & 25 C & 15 \% \\ \hline
    \end{tabular}
    \end{table}
    \item The reset switch resets the timer and clean the console
    \item The frequency switch changes the frequency of the monitoring. The possible frequencies are 500ms, 1s, 2s. The frequency must loop between those values per event occurrence.
    \item The stop switch stops or starts the monitoring of the sensors. The timer is not affected by this functionality.
    \item The display switch displays the first five reading since the stop switch was pressed.
\end{itemize}


At the end of practical 4, you are to submit one .c file and one .pdf file including your answers on Vula. The pdf and C file must have the following format prac3\_STUDNUM001\_STUDNUM002.fileformat