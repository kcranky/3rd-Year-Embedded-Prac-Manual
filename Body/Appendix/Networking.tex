\section{Networking on the Pi}
\label{app:NetworkingOnThePi}
There are many ways to interface with the Pi. This section will cover types of network connectivity.

\subsection{Ethernet}
\label{sec:Connectivity-Ethernet}
This section will assign a static Ethernet address to the Raspberry Pi. This is useful for your first configuration.
\begin{enumerate}
    \item Insert the SD card into a computer and navigate to the BOOT partition
    \item open "cmdline.txt" and append the following:
        \begin{verbatim}ip=192.168.137.15\end{verbatim} 
        This tells the Raspberry Pi to configure the Ethernet port to use the IP address 192.168.137.15
    \item Enable SSH as per Section \ref{sec:SSH}
    \item You need to configure your PC to use the same subnet as the Pi. To do so, see Section \ref{sec:Connectivity-ChangeComputerIP}
\end{enumerate}

\subsection{Changing the IP on your computer}
\label{sec:Connectivity-ChangeComputerIP}
\textbf{Windows}\\
To change the IP of your Ethernet port on Windows 10, complete the following steps:
\begin{itemize}
    \item Right click on your network option in Windows taskbar
    \item Select"Open Network \& Internet Settings", on the lower right hand side of the screen.
    \item Select "Change Adapted Options"
    \item Right click on the Ethernet Connection and select "Properties"
    \item Select "Internet Protocol Version 4 (TCP/IPv4) and click "Properties"
    \item Select "Use the following IP address:" and enter in the following options:\\
            - IP Address: 192.138.137.1\\
            - Subnet Mask: 255.255.255.0
    \item You have successfully changed the IP of the ethernet card on your computer. It is suggested that you now ensure connectivity
\end{itemize}

\subsection{Ensuring connectivity}
\label{sec:Connectivity-EnsuringConnectivity}
Sometimes you may want to debug your connection to the Pi. A fast way to do this is via the \textit{ping} command. \textit{Ping} sends a packet to a particular host (in this case the Pi), and measures the time taken for a response from that host. 

To use the ping command, open a command promt window and type the following:
\begin{verbatim}
    ping 192.168.137.15
\end{verbatim}

If that host is unreachable (the Pi hasn't booted yet or is incorrectly configured), a message will show that the host is unreachable. If everything was correctly configured, you should get
\begin{verbatim}
             Reply from 192.168.137.15: bytes=32 time<1ns TTL=64
\end{verbatim}

This means your Pi and computer are both correctly configured. See section \ref{sec:SSH} for configuring your Pi for SSH access.