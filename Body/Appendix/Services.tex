\section{Services}
\subsection{SSH}
\label{sec:SSH}
\subsubsection{Enabling SSH}
If you have not connected to your Pi and configured it for SSH, you need to do so. SSH is disabled by default on new installations of Raspbian.

To enable SSH, do the following:
\begin{enumerate}
    \item Insert the SD card into a computer 
    \item Navigate to the BOOT partition
    \item Create a file called "ssh" 
    \item Your Pi will enable SSH upon next boot
\end{enumerate}

\subsubsection{Using SSH}
To use SSH on your Pi, you need to connect to the computer to a network. See Section \label{sec:Connectivity} on various ways that can be done (it is suggested to use Ethernet upon first connection).

Once your pi is connected to the computer and you have ensured connection (see Section \ref{sec:Connectivity-EnsuringConnectivity}, use can log in to your Pi via SSH. To do so, do the following:

\begin{itemize}
    \item Open PuTTY
    \item In the "Hostname" field, enter in "192.168.137.15"
    \item Click "Open". A terminal window will be opened. If it is the first time you're SSH'ing into your Pi on this particular computer, you will be asked about the server fingerprint. Click "Yes" to continue.
    \item You will be asked for a username and password. The username is "pi" and the password is "raspberry"
    \item You should now successfully connected to your Raspberry Pi via SSH
\end{itemize}

It is suggested you change the password to keep your implementation secure. You can change your password using the following command:
\begin{verbatim}
    $ passwd
\end{verbatim}

\subsection{VNC}
\label{app:Services-VNC}
In the previous section, control via SSH was introduced. As previously mentioned, the Raspberry Pi can be used as a standalone desktop computer. However, it is a little impractical to carry around a screen and all the other required peripherals when you're working with your Pi. 

There are various options for VNC servers. Raspbian comes installed with Real VNC but it needs to be enabled. Other options, such as tightVNC and ultraVNC also exist and can also be used. 

\begin{enumerate}
    \item Activate Real VNC\\
        \begin{itemize}
            \item Start by connecting to SSH, and opening up raspi-config
        \begin{verbatim}
            $ sudo raspi-config $
        \end{verbatim}
        \item Scroll down using the arrow keys to 5 - Interfacing Options
        \item Scroll down to VNC, and select "Yes" when asked to enable it
        \item Select "Finish"
    \end{itemize}
    
    
    \item Adjust resolution\\
    This can be done in two ways:
    \begin{itemize}
        \item Setting through /boot/config.txt\\
            Edit config.txt and uncomment these lines:
            \begin{verbatim}
                framebuffer_width=1280
                framebuffer_height=720
            \end{verbatim}
        \item On the Pi desktop in VNC\\
            Do this if you have already connected to VNC. This is a little more difficult as it required you to play with windows in order to see the buttons you need.
            \begin{itemize}
                \item Connect to the Pi through VNC. 
                \item In the desktop menu, go to Preferences > Raspberry Pi Configuration and click the "Set Resolution" button. 
                \item Select a more appropriate resolution (1280*720 suggested)
                \item Select "Okay" and then "Okay". You will be asked to reboot your Pi, do so.
            \end{itemize}
            .
    \end{itemize}
    \item Download a viewer\\
    VNC Viewer is available at this URL:\\ \href{https://www.realvnc.com/en/connect/download/viewer/}{https://www.realvnc.com/en/connect/download/viewer/}\\ 
    Download your choice of app (For example the standalone installer or the Chrome App)
    \item Set up the connection
        \begin{itemize}
            \item Open up VNC viewer
            \item Enter the IP of your Pi
            \item Click connect
            \item You will need 
        \end{itemize}
    \item Configure the Pi\\
    Upon first boot to desktop, you will be asked to configure some options on the raspberry pi. Simply hit next/skip through all of them as they will be configured at a later stage.
\end{enumerate}

\subsection{SCP}
\label{sec:SCP}
SCP or "Secure Copy" is a protocol that allows you to transfer files. \newline
To transfer a file from your computer to your Raspberry Pi, run the following command. (This command assumes your Pi is located at IP 192.168.1.15 and the username is still the default "pi"
\begin{verbatim}
    $ scp <compiled_file_name> pi@192.168.1.15:
\end{verbatim}
On Windows, SCP may work out of the box but if it does not, make sure you have the full PuTTY suite installed (see Section \ref{sec:Prereqs}) and run the following:
\begin{verbatim}
    $ pscp <compiled_file_name> pi@192.168.1.15:
\end{verbatim}
After the colon in the above command you can set the directory where you want to copy the file to. If unset, it will simply copy to the home directory. \textbf{TODO Ensure this is true}