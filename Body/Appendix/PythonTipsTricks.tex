\section{Python Tips and Tricks}

\subsection{The RPI.GPIO Library}
The RPi.GPIO library is library used on the Raspberry Pi. 
Documentation for the library can be found here:\\
\href{https://sourceforge.net/p/raspberry-gpio-python/wiki/Home/}{https://sourceforge.net/p/raspberry-gpio-python/wiki/Home/}

It is included in the environment variables by default, so, in order to use it, you can simply just import it:

\begin{verbatim}
    include RPi.GPIO as GPIO
\end{verbatim}


\subsection{Interrupts}
\textbf{TODO}

\subsection{Threading}
\textbf{TODO}

\newpage
\subsection{Python Programming Template}
\begin{lstlisting}
#!/usr/bin/python3
"""
Python Practical Template
Keegan Crankshaw
Readjust this Docstring as follows:
Names: <names>
Student Number: <studnum>
Prac: <Prac Num>
Date: <dd/mm/yyyy>
"""

# import Relevant Librares
import RPi.GPIO as GPIO

# Logic that you write
def main():
    print("write your logic here")


# Only run the functions if 
if __name__ == "__main__":
	# Make sure the GPIO is stopped correctly
	try:
	    while True:
		    main()
	except KeyboardInterrupt:
		print("Exiting gracefully")
		# Turn off your GPIOs here
		GPIO.cleanup()
	except:
		print("Some other error occurred")

\end{lstlisting}