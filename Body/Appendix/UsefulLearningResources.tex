\section{Useful Learning Resources}
% Bash
\subsection{The Unix Shell}
Being able to use the Unix Shell and terminal commands is an invaluable skill, and a requirement for this course. Follow \href{https://swcarpentry.github.io/shell-novice/}{this guide} (https://swcarpentry.github.io/shell-novice/) for learning resources.

% Git
\subsection{Git}
\label{app:Git}
Git is a powerful tool which is essential to managing and tracking your code. It essentially works as a version control system, allowing you to backup changes to your code at defined set points.

Git is a local tool. To use online backups, it's recommended to use a remote (online) repository management tool. GitHub is recommended because, as a student, there are many benefits which you can access. See \href{https://education.github.com/pack}{https://education.github.com/pack} for signing up.

Git can be intimidating in the beginning, but it becomes invaluable as you progress in software development. \href{https://swcarpentry.github.io/git-novice/}{This page} has a great guide you can follow to get well acquainted.

\subsubsection{A Quick Git Get Go}
\paragraph{Creating a GitHub account and configuring your computer appropriately}
\begin{itemize}
    \item Start by creating a GitHub account
    \item Install git on your computer\\
    If you're using a linux based system, this can be as simple as \verb|sudo apt-get install git|\\
    If you're using Windows, you need to download and use an installer.
    \item Once git is installed, run the following commands:\\
    git config --global user.name "Your Name"\\
    git config --glonbal user.email "github email address"
    \item Git is now configured
\end{itemize}

\paragraph{Creating a New Project}\mbox{}\\
Git consists of three primary stages: Untracked, staged and committed. Untracked files are not tracked by the repository. Staged files are files staged for commit but not yet committed. Committed files are "saved" to git.
On your local system:
\begin{itemize}
    \item Create a folder and enter into it
    \item Run \verb|git init|
    \item Create a new text file, for example "test.txt"
    \item Run \verb|git status|
    \item You will see there is an untracked file. Add it to git by running \verb|git add test.txt|
    \item It is possible to add all untracked files by running \verb|git add .|
    \item If you run \verb|git status| again, you will see that "test.txt" has been staged, but not yet committed. Commit test.txt by running \verb|git commit -m "Created test.txt"|. The \verb|-m| flag is to include a git commit message. It's useful to use these messages to explain what has changed in this commit.
\end{itemize}

\paragraph{Linking GitHub and your Local Project}\mbox{}\\
On GitHub, create a new repository and give it a meaningful name and description. Take note of the link (something like https://github.com/\textless username\textgreater /\textless project-name\textgreater .git)

GitHub gives instructions on how to push an existing repository from the command line, but for completeness sake the commands are included here:
\begin{verbatim}
git remote add origin https://github.com/<username>/<project-name>.git
git push -u origin master    
\end{verbatim}

If you refresh the GitHub page, you should now see your files and commits.

\paragraph{Understanding .gitignore}\mbox{}\\
Related SW Carpentry link: \href{https://swcarpentry.github.io/git-novice/06-ignore/index.html}{Ignoring Things}

You may have seen the option to add a .gitginore when creating a new repository on GitHub. This is used to get Git to ignore certain files, such as interim or raw data files. 


\subsection{A Brief Overview of Networks}
\label{app:UnderstandingNetworks}
It is useful to have a basic idea of how networks, IP addresses, and subnets work. For this, it it suggested you read \href{https://support.microsoft.com/en-za/help/164015/understanding-tcp-ip-addressing-and-subnetting-basics}{this article} from Microsoft: \href{https://tinyurl.com/y2z8x9za}{https://tinyurl.com/y2z8x9za}