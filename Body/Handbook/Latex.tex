\section{\LaTeX}
\LaTeX is a great way to write documents, and is required for use in all your documents to be submitted for this course. \LaTeX is better than basic text editors such as Microsoft Word or Google Docs due to the following\footnote{Adapted from \href{https://academia.stackexchange.com/questions/5414/what-are-the-advantages-or-disadvantages-of-using-latex-for-writing-scientific-p}{this stack exchange question}}:
\begin{itemize}
    \item Dealing with mathematical notation.\\
    Layout and entry are generally easier using LaTeX than some other sort of equation editor. Online tools such as \href{http://detexify.kirelabs.org/classify.html}{Detexify} make it very simple to find the symbol you need.
    \item Consistent handling of intra-document references and bibliography.\\
    As of a couple of years ago the major editors still had problems with re-numbering cross-references and bibliography items. This is never a problem with BibTeX or LaTeX.
    \item Separation of content and style.\\
    In principle this means that you can write your document without caring how it is formatted, and at the end of the day wrap it in the style-file provided by the journal publisher before submission to conform to the house style. In practice some of the journal publishers demand special formatting commands that partially moots this process. Furthermore recent versions of Word and LibreOffice Writer, when properly used, should be able to keep track of various levels of section heading separate from the body text, and apply uniform styling to each level. The gap is somewhat closing.
    \item Tables and illustrations.\\
    With online tools such as \href{https://www.tablesgenerator.com/}{Tables Generator}, creating tables in \LaTeX is as simple as copy-pasting data from excel. Images can be inserted exactly where you specify them without worrying about justification or overlay.
\end{itemize}

There are a few difficulties with \LaTeX. These include difficulties with collaborative editing (consider the convenience of Google Docs), spell check (Microsoft Word has a much more advanced spell and grammar check) and ease of use (\LaTeX is technically a "document preparation system" as opposed to a text editor). However, many of these issues are mitigated by the use of an online tool known as \href{https://www.overleaf.com}{Overleaf}. Overleaf provides you with templates, the ability to collaborate, and (thankfully), a spell check function. It runs in browser and doesn't require any installation. If you would like to run an offline version, there are various options, but I suggest using Visual Studio Code with the \href{https://github.com/James-Yu/LaTeX-Workshop/wiki/Install}{Latex Workshop Plugin}.