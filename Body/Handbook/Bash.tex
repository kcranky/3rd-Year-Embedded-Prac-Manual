% Bash
\section{The Unix Shell}
\label{app:UsefulResources}
Being able to use the Unix Shell and terminal commands is an invaluable skill, and a requirement for this course. Follow \href{https://swcarpentry.github.io/shell-novice/}{this guide} (https://swcarpentry.github.io/shell-novice/) for learning resources.

Some useful commands include:
\begin{table}[H]
\centering
\caption{Some useful shell commands}
\label{tbl:commands}
\begin{tabular}{|l|l|}
\hline
\textbf{Command} & \textbf{Use} \\ \hline
ls & \begin{tabular}[c]{@{}l@{}}List current files and folders in directory.\\ ls -al is useful to list everything\end{tabular} \\ \hline
cd \textless{}directory\textgreater{} & \begin{tabular}[c]{@{}l@{}}Change to a specified directory.\\ eg "cd .. " will take you one level up\end{tabular} \\ \hline
ifconfig & Shows current network interfaces and how they are connected \\ \hline
touch \textless{}file\textgreater{} & create \textless{}file\textgreater{} \\ \hline
nano \textless{}file\textgreater{} & Opens \textless{}file\textgreater in the nano text editor \\ \hline
vim \textless{}file\textgreater{} & Opens \textless{}file\textgreater in the vim text editor \\ \hline
mkdir \textless{}dir\textgreater{} & Creates the folder specified in \textless{}dir\textgreater{} \\ \hline
sudo \textless{}cmd\textgreater{} & executes the command \textless{}cmd\textgreater as an administrator \\ \hline
raspi-config &  \begin{tabular}[c]{@{}l@{}}Must be run as administrator\\ Opens up the Raspberry Pi Configuration Tool\end{tabular} \\ \hline
TODO & TODO \\ \hline
\end{tabular}
\end{table}

