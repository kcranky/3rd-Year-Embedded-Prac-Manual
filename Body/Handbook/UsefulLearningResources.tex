% Bash
\section{The Unix Shell}
\label{app:UsefulResources}
Being able to use the Unix Shell and terminal commands is an invaluable skill, and a requirement for this course. Follow \href{https://swcarpentry.github.io/shell-novice/}{this guide} (https://swcarpentry.github.io/shell-novice/) for learning resources.

Some useful commands include:
\textbf{TODO: include useful commands}

% Git
\section{Git}
\label{sec:Git}
Git is a powerful tool which is essential to managing and tracking your code. It essentially works as a version control system, allowing you to backup changes to your code at defined set points.

Git is a local tool. To use online backups, it's recommended to use a remote (online) repository management tool. GitHub is recommended because, as a student, there are many benefits which you can access. See \href{https://education.github.com/pack}{https://education.github.com/pack} for signing up.

Git can be intimidating in the beginning, but it becomes invaluable as you progress in software development. \href{https://swcarpentry.github.io/git-novice/}{This page} has a great guide you can follow to get well acquainted.

\subsection{A Quick Git Get Go}
\subsubsection{Creating a GitHub account and configuring your computer appropriately}
\begin{itemize}
    \item Start by creating a GitHub account
    \item Install git on your computer\\
    If you're using a linux based system, this can be as simple as \verb|sudo apt-get install git|\\
    If you're using Windows, you need to download and use an installer.
    \item Once git is installed, run the following commands:\\
    git config \--\--global user.name "Your Name"\\
    git config \--\--global user.email "github email address"
    \item Git is now configured
\end{itemize}

\subsubsection{Creating a New Project}
Git consists of three primary stages: Untracked, staged and committed. Untracked files are not tracked by the repository. Staged files are files staged for commit but not yet committed. Committed files are "saved" to git.
On your local system:
\begin{itemize}
    \item Create a folder and enter into it
    \item Run \verb|git init|
    \item Create a new text file, for example "test.txt"
    \item Run \verb|git status|
    \item You will see there is an untracked file. Add it to git by running \verb|git add test.txt|
    \item It is possible to add all untracked files by running \verb|git add .|
    \item If you run \verb|git status| again, you will see that "test.txt" has been staged, but not yet committed. Commit test.txt by running \verb|git commit -m "Created test.txt"|. The \verb|-m| flag is to include a git commit message. It's useful to use these messages to explain what has changed in this commit.
\end{itemize}

\subsubsection{Linking GitHub and your Local Project}\mbox{}\\
On GitHub, create a new repository and give it a meaningful name and description. Take note of the link (something like https://github.com/\textless username\textgreater /\textless project-name\textgreater .git)

GitHub gives instructions on how to push an existing repository from the command line, but for completeness sake the commands are included here:
\begin{verbatim}
git remote add origin https://github.com/<username>/<project-name>.git
git push -u origin master    
\end{verbatim}

If you refresh the GitHub page, you should now see your files and commits.

\subsubsection{Understanding .gitignore}
Related SW Carpentry link: \href{https://swcarpentry.github.io/git-novice/06-ignore/index.html}{Ignoring Things}

You may have seen the option to add a .gitginore when creating a new repository on GitHub. This is used to get Git to ignore certain files, such as interim or raw data files. 


\section{\LaTeX}
\LaTeX is a great way to write documents, and is required for use in all your documents to be submitted for this course. \LaTeX is better than basic text editors such as Microsoft Word or Google Docs due to the following\footnote{Adapted from \href{https://academia.stackexchange.com/questions/5414/what-are-the-advantages-or-disadvantages-of-using-latex-for-writing-scientific-p}{this stack exchange question}}:
\begin{itemize}
    \item Dealing with mathematical notation.\\
    Layout and entry are generally easier using LaTeX than some other sort of equation editor. Online tools such as \href{http://detexify.kirelabs.org/classify.html}{Detexify} make it very simple to find the symbol you need.
    \item Consistent handling of intra-document references and bibliography.\\
    As of a couple of years ago the major editors still had problems with re-numbering cross-references and bibliography items. This is never a problem with BibTeX or LaTeX.
    \item Separation of content and style.\\
    In principle this means that you can write your document without caring how it is formatted, and at the end of the day wrap it in the style-file provided by the journal publisher before submission to conform to the house style. In practice some of the journal publishers demand special formatting commands that partially moots this process. Furthermore recent versions of Word and LibreOffice Writer, when properly used, should be able to keep track of various levels of section heading separate from the body text, and apply uniform styling to each level. The gap is somewhat closing.
    \item Tables and illustrations.\\
    With online tools such as \href{https://www.tablesgenerator.com/}{Tables Generator}, creating tables in \LaTeX is as simple as copy-pasting data from excel. Images can be inserted exactly where you specify them without worrying about justification or overlay.
\end{itemize}

There are a few difficulties with \LaTeX. These include difficulties with collaborative editing (consider the convenience of Google Docs), spell check (Microsoft Word has a much more advanced spell and grammar check) and ease of use (\LaTeX is technically a "document preparation system" as opposed to a text editor). However, many of these issues are mitigated by the use of an online tool known as \href{https://www.overleaf.com}{Overleaf}. Overleaf provides you with templates, the ability to collaborate, and (thankfully), a spell check function. It runs in browser and doesn't require any installation. If you would like to run an offline version, there are various options, but I suggest using Visual Studio Code with the \href{https://github.com/James-Yu/LaTeX-Workshop/wiki/Install}{Latex Workshop Plugin}.
