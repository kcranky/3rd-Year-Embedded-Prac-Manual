\section{Python Tips and Tricks}
\label{app:Python}
This chapter contains some useful tips and tricks to writing good Python for embedded systems. You will also find the template you are required to use for your practicals, as well as how to make use of good practices, such as debouncing, or making use of the Raspberry Pi's multicore architecture by implementing threading.

Python, while not as powerful as C, is quickly becoming a common choice for embedded systems developers due to its ease of use. \footnote{See \href{https://spectrum.ieee.org/at-work/innovation/the-2018-top-programming-languages}{https://spectrum.ieee.org/at-work/innovation/the-2018-top-programming-languages}}. 

\subsection{The RPI.GPIO Library}
The RPi.GPIO library is library used on the Raspberry Pi. 
Documentation for the library can be found here:\\
\href{https://sourceforge.net/p/raspberry-gpio-python/wiki/Home/}{https://sourceforge.net/p/raspberry-gpio-python/wiki/Home/}

It is included in the environment variables by default, so, in order to use it, you can simply just import it:

\begin{verbatim}
    include RPi.GPIO as GPIO
\end{verbatim}

\subsection{Python Programming Template}
\begin{lstlisting}
#!/usr/bin/python3
"""
Python Practical Template
Keegan Crankshaw
Readjust this Docstring as follows:
Names: <names>
Student Number: <studnum>
Prac: <Prac Num>
Date: <dd/mm/yyyy>
"""

# import Relevant Librares
import RPi.GPIO as GPIO

# Logic that you write
def main():
    print("write your logic here")


# Only run the functions if 
if __name__ == "__main__":
	# Make sure the GPIO is stopped correctly
	try:
	    while True:
		    main()
	except KeyboardInterrupt:
		print("Exiting gracefully")
		# Turn off your GPIOs here
		GPIO.cleanup()
	except:
		print("Some other error occurred")

\end{lstlisting}

\subsection{Interrupts}
Adding an Interrupt in Python is as simple as:
\begin{verbatim}
    GPIO.add_event_detect(BTN_B, GPIO.RISING, method_on_interrupt)
\end{verbatim}

