\section{Setting up your Pi}
In order to use the Pi you need to install an operating system on it and setup its networking.  The following will lead you through installing Raspbian as an operating system.  Raspbian  is a fork of the Debian distribution of the open source operating system Linux.  You will then configure the networking settings in Raspbian to allow you to access the Pi remotely using SSH.  

It is assumed that the user is running Windows, but the process will be similar for users of any operating system. It is recommended that during this course users gain familiarity with a Linux based OS such as Ubuntu

\subsection{Prerequisites}
\label{sec:Prereqs}
Download the following software applications and install them as indicated on their webpages
\textbf{TODO: These resources are also available on Vula.}
\begin{itemize}
    \item Latest Rasbian Stretch with Desktop and Recommended Software image, available at\\ \href{https://www.raspberrypi.org/downloads/raspbian}{Raspberry Pi Foundation Website}
    \item PuTTY (the full suite) available at\\ \href{https://www.chiark.greenend.org.uk/~sgtatham/putty/latest.html}{https://www.chiark.greenend.org.uk/~sgtatham/putty/latest.html}. \footnote{Some versions of Windows have the required SSH and SCP packages, but PuTTY has many other tools that are useful.}
    \item SDFormatter, available at\\ \href{https://www.sdcard.org/downloads/formatter_4/eula_windows/index.html}{https://www.sdcard.org/downloads/formatter\_4/eula\_windows/index.html}
    \item Etcher, an image writing tool, available at\\
    \href{https://www.balena.io/etcher/}{https://www.balena.io/etcher/}
\end{itemize}

\subsection{Install and configure Raspbian}
\begin{enumerate}
    \item Insert the SD card into the computer. If you do not have a reader available, speak to a friend or tutor who does.
    \item Write image to SD card\\
        Open Etcher. Select the downloaded zip image, and the SD card, and format. At the end of format, it may read that it failed,  but don't worry. Upon completion, Windows will try to mount partitions on the SD card that it can't read. Just press "Cancel" and then "OK" to the dialog boxes that pop up. The boot partition is the only partition we will be dealing with.
    \item Boot\\
    Insert SDCard into Pi and Plugin a screen and keyboard. If you do not have a keyboard, mouse and screen available, refer to the headless installation below.
    \item Configure\\
    For practicals you will need to enable a few services on the Pi.
    \begin{itemize}
        \item Open a terminal. You can either click the icon or press Ctrl+Alt+T
        \item Run \$ sudo raspi-config
        \item Scroll down to "interfacing Options"
        \item Enable SSH, VNC, SPI, I2C and Serial
    \end{itemize}
\end{enumerate}

\subsection{Headless Installation}
Headless mode on the Raspberry Pi refers to using it without direct user input and output (essentially no screen, mouse or keyboard connected directly to it). This is how all of the practicals will be conducted, as it is how most IoT devices are configured (over a network).

To use the RPi in headless mode, you at least need to enable SSH. Refer to Section \ref{sec:SSH} for instructions on how to do so. Once you have enabled SSH, be sure to run \$sudo raspi-config and enable VNC, SPI, I2C and Serial.

\subsection{Testing}
Once you've completed the installation process, including all relevant network configurations and SSH, be sure to test your connection and configuration before the first lab by logging in to the Pi through SSH or VNC.